\chapter*{Introduction}
\addcontentsline{toc}{chapter}{Introduction}

The process of designing electrical circuits consists of several stages, starting with detailed specification from the customer that provides all necessary requirements,
and ending with a working prototype of the final product. Intermediate stages include prototyping circuits using a construction base commonly referred to as breadboard, which is is often
very slow and for complex integrated circuits even impossible. The task has been made easier with the invention of electrical circuit simulation programs, which allow quick circuit prototyping
without the need of soldering iron.

Today's computers have enough computing capacity that we can consider using evolutionary algorithms to evolve electrical circuit parameters. We can even
use techniques of genetic programming for evolving the circuit's topology \todo{reference some relevant article?}, which is yet another reason why
electrical circuit simulators are very valuable in the electro-technical industry  \todo{Note: inspired from Inside SPICE}.

\section{The Berkeley SPICE}
One of the most successful circuit simulators was the SPICE program developed at EECS Department of the University of California, Berkeley. The name stands for \textbf{S}imulation \textbf{P}rogram with \textbf{I}ntegrated \textbf{C}ircuit \textbf{E}mphasis. The original SPICE1 program was implemented in \todo{Fortran} and was released in 1971. It's popoularity quickly rose and few years later, Berkeley released SPICE2 with many performance improvements and model enhancements.

During the 1980s, Berkeley decided to rewrite SPICE2 into the C programming language to make it better suited to --- at that time increasingly more popular --- UNIX-based operating systems \cite{spice_book}. The resulting SPICE3 was to be superset of SPICE2, but the development was in the end left to handful of students due to the limited financial resources. The first release of SPICE3 was very buggy and was not even backward compatible with SPICE2. It was mainly due to the incompatibility that SPICE3 did not replace SPICE2, because hundreds of commonly used macro-models would have to be rewritten before they could be used in SPICE3 \cite{inside_spice}.

\section{SPICE3 and K\&R C}
The SPICE3 was written in now non-standard C K\&R norm, because the development started years before the first ANSI norm was released in 1990. Due to the nature of the C language, many parts of the simulator are very fragile. Consider following code snippet from last SPICE3 version source code. It contains a function that loads instances of voltage controlled current source devices to the equation matrix. On the very first line of the function, the \todo{code} inModel parameter is cast to pointer to concrete type of the device.

\todo{include source paths?}
\begin{code}
	/*ARGSUSED*/
	int
	VCCSload(inModel,ckt)
	GENmodel *inModel;
	CKTcircuit *ckt;
	/* actually load the current values into the 
	* sparse matrix previously provided 
	*/
	{
		register VCCSmodel *model = (VCCSmodel *)inModel;
		register VCCSinstance *here;
		
		/*  loop through all the source models */
		for( ; model != NULL; model = model->VCCSnextModel ) {
			
			/* loop through all the instances of the model */
			for (here = model->VCCSinstances; here != NULL ;
			here=here->VCCSnextInstance) {
				
				*(here->VCCSposContPosptr) += here->VCCScoeff ;
				*(here->VCCSposContNegptr) -= here->VCCScoeff ;
				*(here->VCCSnegContPosptr) -= here->VCCScoeff ;
				*(here->VCCSnegContNegptr) += here->VCCScoeff ;
			}
		}
		return(OK);
	}
\end{code}

This function is one example of many functions that must exist for each device to be used in SPICE3. Other such functions include methods for model updating and data printing. Such an arrangement can be a source of many hard-to-debug bugs when attempting to extend the circuit simulator.

%
%\begin{code}
%typedef struct sGENmodel {       /* model structure for a resistor */
%	int GENmodType; /* type index of this device type */
%	struct sGENmodel *GENnextModel; /* pointer to next possible model in 
%	* linked list */
%	GENinstance * GENinstances; /* pointer to list of instances that have this
%	* model */
%	IFuid GENmodName;       /* pointer to character string naming this model */
%} GENmodel;
%
%//....
%
%typedef struct sVCCSmodel {       /* model structure for a source */
%	int VCCSmodType;    /* type index of this device type */
%	struct sVCCSmodel *VCCSnextModel;    /* pointer to next possible model 
%	* in linked list */
%	VCCSinstance * VCCSinstances;    /* pointer to list of instances 
%	* that have this model */
%	IFuid VCCSmodName;       /* pointer to character string naming this model */
%} VCCSmodel;
%
%\end{code}
%

%Since the release of SPICE3, many open-source and commercial circuit simulators have been developed. Examples include PSpice, HSpice and LTspice. Most of these modern simulators accept circuit description in a format that is similar the original SPICE program and can therefore make use of the vast amount of existing macro-models.
%
%The main goal of this thesis is implementation of an extensible SPICE-like simulation library in a modern programming language to be used in applications that require circuit simulations. \todo{this seems out of place.}

\section{Nonconvergence due to insufficient numeric precision}

Most of the common circuit simulators use precision type \todo{code environment} double with approximately 15 decimal significant digits precision. This leads to truncation errors while finding the solution of circuit equations with coefficients that differ in more than 15 orders of magnitude. The truncation errors in turn lead to noise in simulator output or in the worst case to nonconvergence of the solution.

Coefficient differences of this magnitude may commonly occur when resistors with very small resistance values are used. Even though this is generally discouraged and avoided for exactly these reasons, such resistors may result from macromodel \todo{use macromodel, or macro-model consistently} construction. \todo{cite } (https://www.edn.com/design/analog/4418707/1/Extended-precision-simulation-cures-SPICE-convergence-problems).
\todo{example of macromodel construction?}

Following figure shows an example circuit for which the double precision is insufficient.

\todo{example circuit figure + plot of expected and actual values}

This thesis therefore also investigates the possibility of using extended precision technique known as double-double in circuit simulations. This method uses two floating point variables $h, l$, one containing the upper half of significant digits and the other the lower half, to represent a single value $h + l$. As opposed to special types like \todo{code} long double in C++, which are implementation defined and often have higher precision only on specialized hardware, this approach uses standard operations available on all common processors. \todo{mention that value is represented as sum of components?}

\section{Goals of this Thesis}
The goal of this thesis is to reimplement a certain subset of SPICE's functionality using C\# .NET. Implementation will be divided into two principal components -- a reusable library for simulating electrical circuicts, and a console application capable of reading SPICE like netlists \todo{just a subset?} and perform analyses using said library.

Emphasis will be given mainly to extensibility and re-usability of the core simulation library. The library will expose an interface for adding new circuit devices and even new analysis types, without the need for the library's source code modification. Therefore, the standalone library could be used in any application that needs simulate electrical circuits.

This thesis will also investigate higher precision technique double-double and it's appropriateness for circuit simulation with respect to runtime and precision.

\subsection{Supported Analysis types}
The library will implement the following analysis types
\begin{itemize}
	\item DC Operating Point Analysis
	\item Transient Analysis
\end{itemize}

\subsection{Supported Devices}
The library will implement at least the following device types for above mentioned analyses.
\begin{itemize}
	\item Ideal resistor
	\item Ideal voltage source
	\item Ideal current source
	\item Ideal inductor
	\item Ideal capacitor
	\item SPICE-like diode model
\end{itemize}

